\chapter{Chapter Title}

An introduction here (with a cite from bibliography, like this: \cite{greenwade93}).
%  ----
\section{Type of texts}
\subsection{Normal text}
\lipsum[1]

\subsection{Normal text}
\textbf{\lipsum[1]}

\subsection{Italic text}
\textit{\lipsum[1]}

\subsection{Mono space text}
\subsubsection{(usually I use this type of format to refer me to code)}
\texttt{Here is some text. Pay attention to this type, because It can go 
\\ out of bounds! Usually you solve it with a double back-slash, to 
\\ go to the new line.}
%  ----

%  ----
\section{Lists}
\subsection{Unbulleted lists}
\begin{itemize}
    \item \lipsum[1]
    \item \lipsum[1]
    \item \lipsum[1]
\end{itemize}

\subsection{Bulleted lists}
\begin{enumerate}
    \item \lipsum[1]
    \item \lipsum[1]
    \item \lipsum[1]
\end{enumerate}
%  ----

%  ----
\section{Images}
Random Duolingo Image example:
\begin{figure}[H]
\centering
    \includegraphics[scale=0.1]{images/duolingo.png}
    \caption{Caption example.}
    \label{fig:duolingo}
\end{figure}

In the text, you reference an image in this way: \ref{fig:duolingo}.
%  ----

%  ----
\section{Tables}
Example of table that can even be split into several pages if it's too long:
\begin{longtable}[H]{| l | l |}
    \hline
     \rowcolor[HTML]{F87C58}\textbf{Head 1} & \textbf{Head 2}\\
    \hline
    \endfirsthead
    
    \texttt{Text} & \texttt{Text}\\
    \hline
    \texttt{text} & \texttt{Text}\\
    \hline
    \texttt{Text} & \texttt{Text}\\
    \hline
    \texttt{text} & \texttt{Text}\\
    \hline
    \texttt{Text} & \texttt{Text}\\
    \hline
    \texttt{text} & \texttt{Text}\\
    \hline
    \texttt{Text} & \texttt{Text}\\
    \hline
    \texttt{text} & \texttt{Text}\\
    \hline
    \texttt{Text} & \texttt{Text}\\
    \hline
    \texttt{text} & \texttt{Text}\\
    \hline
    \texttt{Text} & \texttt{Text}\\
    \hline
    \texttt{text} & \texttt{Text}\\
    \hline
    \texttt{Text} & \texttt{Text}\\
    \hline
    \texttt{text} & \texttt{Text}\\
    \hline
    \texttt{Text} & \texttt{Text}\\
    \hline
    \texttt{text} & \texttt{Text}\\
    \hline
    \texttt{Text} & \texttt{Text}\\
    \hline
    \texttt{text} & \texttt{Text}\\
    \hline
    \texttt{Text} & \texttt{Text}\\
    \hline
    \texttt{text} & \texttt{Text}\\
    \hline
    \caption{Table caption.}
\end{longtable}
%  ----

%  ----
\section{Code}
Finally, here we are with code! The following examples are with typescript and HTML with Angular syntax, colored using the \textbf{minted} package.

\begin{code}
    \inputminted{typescript}{listings/ts-example.ts}
    \caption{Caption for the typescript example.}
\end{code}
\vspace{4mm}

\begin{code}
    \inputminted{ng2}{listings/html-example.html}
    \caption{Caption for the HTML with Angular syntax example.}
\end{code}
\vspace{4mm}
%  ----
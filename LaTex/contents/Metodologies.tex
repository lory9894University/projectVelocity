\chapter{Greenfield vs Brownfield software development}
\label{ch:greenfield_vs_brownfield}
La scrittura di software nell'ambito di un progetto complesso richiede la scelta di diverse metodologie di sviluppo.
Queste metodologie influenzano sia la struttura del progetto e come il software viene sviluppato (ad esempio \textit{DevOps}) 
che il modo in cui il team di sviluppo lavora (ad esempio \textit{Agile}).
La scelta di applicare una metodologia piuttosto che un'altra (o anche diverse metodologie in parallelo) dipende da diversi fattori, tra cui la complessità del progetto,
la dimensione del team di sviluppo e l'organizzazione dello stesso.\\
Nell'ambito della scrittura di un software che va a sostituire un software già esistente, si possono distinguere due metodologie di sviluppo: \texttt{Greenfield} e \texttt{Brownfield}.

\section{Greenfield software development}
\label{sec:greenfield}
Lo \textit{sviluppo Greenfield} è una metodologia di sviluppo software che si basa sulla creazione di un nuovo software da zero, senza alcun vincolo o dipendenza da software preesistenti.
Il team di sviluppo ha quindi la possibilità di scegliere le tecnologie e le metodologie di sviluppo che ritiene più adatte al progetto, e di sviluppare il software come preferisce.
Questo approcio presenta diversi vantaggi, tra cui:
\begin{itemize}
    \item \textbf{Tecnologie più efficaci}: Potendo scegliere le tecnologie da utilizzare, il team di sviluppo può scegliere le migliori tecnologie disponibili per il progetto. 
    In generale tecnologie più recenti e avanzate rispetto a quelle utilizzate in software preesistenti permettono la scrittura di codice più efficiente e performante.  
    \item \textbf{Libertà di scelta}: lo sviluppo di un nuovo software da zero è generalmente più semplice rispetto alla modifica di un software preesistente, dato che presnta pochi vincoli.
    Non essendoci vincoli software, nozionali o di business, il team di sviluppo ha la possibilità di sviluppare il software come preferisce.
    \item \textbf{Software più strutturato}: la possibilità di sviluppare il software da zero permette di creare un software più strutturato e organizzato, quindi più manutenibile ed estendiile.
    L'estensione di un software \textit{legacy} in generale porta a un software più disorganizzato e difficile da mantenere.
\end{itemize}
Tuttavia, lo sviluppo di un software da zero presenta anche diversi svantaggi:
\begin{itemize}
    \item \textbf{Tempo}: Lo sviluppo di un software da zero richiede tempo, in quanto è necessario sviluppare tutte le funzionalità.
    Deve essere considerato anche il tempo necessario alla formazione del team di sviluppo sulle eventuali nuove tecnologie scelte.
    \item \textbf{Costi}: Lo sviluppo di un software da zero richiede maggiori risorse, necessitando di maggiore tempo e di eventuali nuove tecnologie.
    (costi dovuti a licenze software, formazione del personale, nuove infrastrutture, etc ...)
    \item \textbf{Rischio}: Dato che le tecnologie e le metodologie di sviluppo sono innovative, esiste il rischio che queste non siano le migliori per il progetto.
    Il team di sviluppo potrebbe non avere esperienza con le nuove tecnologie, e quindi potrebbe rendersi conto di questo problema molto avanti nello sviluppo del software.
    \item \textbf{Attuabilità}: Spesso lo sviluppo \textit{Greenfield} di un software non è attuabile, in quanto il software preesistente è troppo complesso o troppo integrato con altre parti del sistema.
\end{itemize}

\section{Brownfield software development}
\label{sec:brownfield}
Lo \textit{sviluppo Brownfield} è una metodologia di sviluppo software che si basa sulla modifica graduale di un software preesistente, integrando nuove funzionalità o modificando quelle esistenti.
Questa è la situazione più comune in ambito lavorativo, in quanto la maggior parte delle aziende possiede software preesistenti che necessitano di manutenzione o estensione.
Lo sviluppo \textit{Brownfield} presenta diversi vantaggi:
\begin{itemize}
    \item \textbf{Riduzione dei costi}: Estendere un software preesistente è generalmente più economico e richiede meno tempo rispetto allo sviluppo di un software da zero.
    parte del software è già sviluppata e testata, quindi non è necessario svilupparla nuovamente.
    \item \textbf{Maggiore controllo}: Essendo la transizione graduale i nuovi componenti possono essere testati e integrati in modo più controllato ed incrementale.
    \item \textbf{Progetto già strutturato}: Essendoci una base già esistente non bisogna preoccuparsi di implementare nuove tecnologie o sviluppare nuove architetture.
    Questo riduce i rischi ed l'impiego di tempo necessario all' introduzione di nuove tecnologie.
\end{itemize} 
Naturalmente, lo sviluppo di un software preesistente presenta anche diversi svantaggi:
\begin{itemize}
    \item \textbf{Vincoli}: Lo sviluppo di un software preesistente presenta diversi vincoli.
    Questi vincoli possono limitare le scelte del team di sviluppo, ad esempio limitando le tecnologie che possono essere utilizzate.
    \item \textbf{Complessità}: Lo sviluppo di un software preesistente è generalmente più complesso rispetto allo sviluppo di un software da zero, in quanto bisogna implementare software che sia
    compatibile con il software preesistente. Richiede una maggiore competenza di business ed una maggiore comprensione del software preesistente.
    \item \textbf{Manutenibilità}: Lo sviluppo di un software preesistente può rendere il software più disorganizzato e difficile da mantenere, in quanto le modifiche possono portare a \textit{spaghetti code}.
\end{itemize}
Quasi sempre nella realtà ci si trova obbligati ad utilizzare un approccio \textit{Brownfield}, dato che la riscrittura di un software da zero è spesso troppo onerosa.

\section{Approcio Brownfield per la transizione a microservizi}
\label{sec:brownfield_microservices}
Nel caso della transizione da un'architettura monolitica ad un'architettura a microservizi, come nel caso di questo progetto, l'approcio \textit{Brownfield} è molto comune.
Al posto di riscrivere tutto il sistema basandolo su microservizi, si preferisce migrare gradualmente il sistema esistente verso un'architettura a microservizi,
integrando di volta in volta i nuovi servizi implementati con il sistema esistente.
Questo approccio si sviluppa in diverse fasi:
In primo luogo il sistema esistente viene analizzato, andado ad identificare le varie responsabilità basandosii sulla \textit{business logic}.
Avendo identificato le varie responsabilità si procede iterativamente:
\begin{enumerate}
    \item \textbf{Categorrizazione}: Si categorizzano le varie responsabilità in base alla loro complessità e alla loro dipendenza con altre parti del sistema.
    \item \textbf{Composizione}: Per ogni "categoria" si sviluppa un microservizio che si prenda carico di tutte le responsabilità della categoria.
    \item \textbf{Integrazione}: Il microservizio sviluppato viene integrato con il sistema esistente, sostituendo la parte di sistema che si occupava delle responsabilità.
    La parte di sistema sostituita viene generalmente mantenuta, perchè spesso non è possibile estrarla dal sistema. Quindi è molto comune che per un periodo di tempo
    limitato coesistano due sistemi che si occupano delle stesse responsabilità (un microservizio ed una parte del sistema monolitico).
\end{enumerate}
Una volta eseguite queste fasi per tutte le responsabilità del sistema, il sistema monolitico è stato completamente sostituito da un sistema a microservizi e può essere rimosso.
Questo approccio permette una transizione graduale e controllata, riducendo i rischi e permettendo di mantenere il sistema completamente funzionante durante la transizione.
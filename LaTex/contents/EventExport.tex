\chapter{Microservizio EventExport}
\label{cap:MicroservizioEventExport}

\section{Panoramica del microservizio}
\label{sec:ScopoDelMicroservizio}
Lo scopo del microservizio \textit{EventExport} è quello di elaborare dei \textbf{Business Event}, che arrivano dal microservizio \textit{EventEngine} 
(sezione \ref{subsubsec:event_engine}) e di comunicarli al cliente.\\\\

\textbf{EventExport} è configurabile in maniera differente per ogni cliente, di modo che ad un cliente vengano inviati solo alcuni tipi di eventi.
Ad esempio, un cliente potrebbe essere interessato solo agli eventi di creazione di un ordine, mentre un altro potrebbe voler ricevere tutti gli aggiornamenti
riguardo alla posizione dell'ordine da lui effettuato.
Questa configurazione è gestita tramite una tabella \textit{LogicDeterminationEvent} che contiene le regole di filtro per ogni cliente.
%TODO: aggiungere immagine della tabella LogicDeterminationEvent (prendila dalla documentazione o da DBeaver)

%TODO: aggiungere l'immagine della documentazione, quella dove si vede in che posto si colloca EventExport rispetto agli altri microservizi (? la metto ?)

Step 1: Consumere il \textit{Signaling Topic} (topic Kafka) applicando un filtro basato su una tabella di configurazione.
Step 2: controllare se l’evento deve essere inviato, non inviare eventi già inviati a meno di modifiche su campi significati
Step 3: in base alla configurazione creare XML con i dati richiesti dal cliente, saranno definiti dei template per compilare le varie sezioni informative in base a una configurazione
Step 4: inviare a Sterling tramite API di prodotto gli XML prodotti